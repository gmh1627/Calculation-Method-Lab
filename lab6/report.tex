\documentclass{ctexart}
\usepackage{graphicx}
\usepackage{caption}
\usepackage{float}
\usepackage{amsmath}
\usepackage{fancyhdr}
\usepackage{xunicode-addon}
\usepackage{booktabs}
\usepackage[a4paper,hmargin=1.25in,vmargin=1in]{geometry}
% !TeX program = xelatex
\title{\begin{figure}[H]
	\centering 
	\includegraphics[height=7cm,width=14cm]{E:/Pictures/中科大.jpg}
	\end{figure}\Huge\textbf{Lab 6}\\\huge{FFT与IFFT}}
\date{}
\punctstyle{banjiao} 
\pagestyle{fancy}
	\fancyhead[C]{\LARGE\textbf{Lab 6}}
	\fancyhead[L]{}
	\fancyhead[R]{}
	\fancyfoot[C]{\thepage}
\begin{document}
	\maketitle
	\thispagestyle{empty}
	
	\[\makebox{\Large{姓名:\underline{\makebox[5cm]{高茂航}}}}\]
	
    \[\makebox{\Large{学号:\underline{\makebox[5cm]{PB22061161}}}}\]
	
	$$\makebox{\Large{日期:\underline{\makebox[5cm]{2024.5.26}}}}$$
	
	\clearpage

	\pagenumbering{arabic}
	\section{Problem Descriptions}
	通过快速傅里叶变换与快速傅里叶逆变换实现对给定函数的 Fourier 分析以及重建。
	\section{Analysis and Algorithms} 
	\subsection{Algorithm 1 FFT}

$n\leftarrow length[f]$ 

$\textbf{if }n= = 1$

$\quad\quad\textbf{then return }f$

$\textbf{end if}$

$\omega_n\leftarrow e^{\boldsymbol{i}2\pi/n}$

$\omega\leftarrow1$

$\mathbf{f}^{0}\leftarrow(f_{0},f_{2},\ldots,f_{n-2})$

$\mathbf{f}^{1}\leftarrow(f_{1},f_{3},\ldots,f_{n-1})$

$\mathbf{g}^0\leftarrow$FFT$(\mathbf{f}^0)$

$\mathbf{g}^1\leftarrow$FFT$(\mathbf{f}^1)$ 

$\textbf{for }k\leftarrow 0$ to $n/ 2- 1\textbf{do}$

$\quad\quad\mathbf{g_k}\leftarrow\mathbf{g}_k^0+\omega\mathbf{g}_k^1$

$\quad\quad\mathbf{g_{k+n/2}}\leftarrow\mathbf{g}_k^0-\omega\mathbf{g}_k^1$

$\quad\quad\omega\leftarrow\omega\omega_n$

\textbf{end for}

\textbf{return g}
	\subsection{Algorithm 2 IFFT}
	\subsubsection{方法1}

$n\leftarrow length[f]$ 

$\textbf{if }n= = 1$

$\quad\quad\textbf{then return }f$

$\textbf{end if}$

$\omega_n\leftarrow e^{\boldsymbol{-i}2\pi/n}$

$\omega\leftarrow1$

$\mathbf{f}^{0}\leftarrow(f_{0},f_{2},\ldots,f_{n-2})$

$\mathbf{f}^{1}\leftarrow(f_{1},f_{3},\ldots,f_{n-1})$

$\mathbf{g}^0\leftarrow$IFFT$(\mathbf{f}^0)$

$\mathbf{g}^1\leftarrow$IFFT$(\mathbf{f}^1)$ 

$\textbf{for }k\leftarrow 0$ to $n/ 2- 1\textbf{do}$

$\quad\quad\mathbf{g_k}\leftarrow\mathbf{g}_k^0+\omega\mathbf{g}_k^1$

$\quad\quad\mathbf{g_{k+n/2}}\leftarrow\mathbf{g}_k^0-\omega\mathbf{g}_k^1$

$\quad\quad\omega\leftarrow\omega\omega_n$

\textbf{end for}

$\mathbf{g} = \mathbf{g}/2$

\textbf{return g}
\subsubsection{方法2}
$\mathbf{f_{temp}}\leftarrow(f_{0},f_{n-1},f_{n-2}\ldots,f_{2})$

$\mathbf{g}\leftarrow$FFT$(\mathbf{f_{temp}})$

$\mathbf{g} = \mathbf{g}/n$

$\textbf{return g}$
	\section{Results}
	\begin{figure}[H]
		\centering 
		\includegraphics[height=1.5cm,width=14cm]{7.png}
		\caption{}
	\end{figure}
	\begin{figure}[H]
		\centering 
		\includegraphics[height=4.5cm,width=14cm]{8.png}
		\caption{}
	\end{figure}
	\begin{figure}[H]
		\centering 
		\includegraphics[height=7cm,width=14cm]{1.png}
		\caption{}
	\end{figure}
	\begin{figure}[H]
		\centering 
		\includegraphics[height=4.5cm,width=14cm]{9.png}
		\caption{}
	\end{figure}
			\begin{figure}[H]
				\centering 
				\includegraphics[height=7cm,width=14cm]{2.png}
				\caption{}
			\end{figure}
				\begin{figure}[H]
					\centering 
					\includegraphics[height=7cm,width=14cm]{3.png}
					\caption{}
				\end{figure}
					\begin{figure}[H]
						\centering 
						\includegraphics[height=7cm,width=14cm]{4.png}
						\caption{}
					\end{figure}
						\begin{figure}[H]
							\centering 
							\includegraphics[height=7cm,width=14cm]{5.png}
							\caption{}
						\end{figure}
							\begin{figure}[H]
								\centering 
								\includegraphics[height=7cm,width=14cm]{6.png}
								\caption{只取前25\%频率的结果}
							\end{figure}
		\section{Conclusion}
		1. 采样数目为128时,抽样和重建的结果更加接近原函数,且FFT后频域上模长不为0 的频率与 采样数目为16时相同,但模长更大;
		
		2.对于f2,分析去掉高频系数后IFFT的结果明显偏离于原来的抽样结果,因为去掉高频系数时会造成频率的损失。
		
		
    \end{document}