\documentclass{ctexart}
\usepackage{graphicx}
\usepackage{caption}
\usepackage{float}
\usepackage{amsmath}
\usepackage{fancyhdr}
\usepackage{xunicode-addon}
\usepackage{booktabs}
\usepackage[a4paper,hmargin=1.25in,vmargin=1in]{geometry}
% !TeX program = xelatex
\title{\begin{figure}[H]
	\centering 
	\includegraphics[height=7cm,width=14cm]{E:/Pictures/中科大.jpg}
	\end{figure}\Huge\textbf{Lab 1}\\\huge{圆形镜面成像问题}}
\date{}
\punctstyle{banjiao} 
\pagestyle{fancy}
	\fancyhead[C]{\LARGE\textbf{Lab 1}}
	\fancyhead[L]{}
	\fancyhead[R]{}
	\fancyfoot[C]{\thepage}
\begin{document}
	\maketitle
	\thispagestyle{empty}
	
	\[\makebox{\Large{姓名:\underline{\makebox[5cm]{高茂航}}}}\]
	
    \[\makebox{\Large{学号:\underline{\makebox[5cm]{PB22061161}}}}\]
	
	$$\makebox{\Large{日期:\underline{\makebox[5cm]{2024.3.8}}}}$$
	
	\clearpage

	\pagenumbering{arabic}

	\section{Algorithm Description}
	设$T(cos\theta,sin\theta)$,则有
	$$
	\begin{aligned}
	PT+QT &= \sqrt{(P_x-\cos\theta)^2+\sin^2\theta}+\sqrt{(Q_x-\cos\theta)^2+(Q_y-\sin\theta)^2} \\
	&= \sqrt{P_x^2-2P_x\cos\theta+1}+\sqrt{Q_x^2+Q_y^2+1-2Q_x\cos\theta-2Q_y\sin\theta}
	\end{aligned}
	$$
		由费马原理,光线沿$PT+QT$最短的路径传播,因此只需对上式求导求极小值点。关于$\theta$求导得
		$$\frac{P_xsin\theta}{\sqrt{P_x^2-2P_xcos\theta+1}}+\frac{Q_xsin\theta-Q_ycos\theta}{\sqrt{Q_x^2+Q_y^2+1-2Q_xcos\theta-2Q_ysin\theta}}$$
		故只需用二分法解非线性方程
		$$P_xsin\theta\sqrt{Q_x^2+Q_y^2+1-2Q_xcos\theta-2Q_ysin\theta}+(Q_xsin\theta-Q_ycos\theta)\sqrt{P_x^2-2P_xcos\theta+1}=0$$
		$$T_x=cos\theta$$
		$$T_y=sin\theta$$
		由对称性易知
	$$R_x=\frac{2Q_y-Q_xtan\theta -kQ_x +k(2T_x - Q_x) - 2T_y}{k-tan\theta}$$
	$$R_y=Q_y-(Q_x - R_x)\theta$$
	
\section{Results}
$$P = (-2, 0), Q = (-1, 1):T = (-0.885670, 0.464316), R = (-0.380057, 0.674993)$$

$$P = (-10, 0), Q = (-2, 1): T = (-0.959312, 0.282350), R = (0.304214, 0.321811)$$

$$P = (-1.000001, 0), Q = (-2, 2):T = (-1.000000 , 0.000002) , R = (0.000007 , 1.999996)$$

$$P = (-2, 0), Q = (-1, 0.000001):T = (-1.000000 , 0.000001) , R = (-1.000000 , 0.000001)$$

$$P = (-2.33, 0), Q = (-3, 1):T = (-0.989279 , 0.146038) , R = (1.182424 , 0.382590)$$

$$P = (-3, 0), Q = (-1, 0.5):T = (-0.922615 , 0.385721) , R = (-0.786920 , 0.410917)$$

$$P = (-3, 0), Q = (-2, 10):T = (-0.827028 , 0.562160) , R = (8.380296 , 2.944148)$$

$$P = (-3, 0), Q = (-3, 1):T = (-0.987408 , 0.158192) , R = (1.187435 , 0.329136)$$

$$P = (-10, 0), Q = (-2, 1):T = (-0.959312 , 0.282350) , R = (0.304214 , 0.321811)$$

$$P = (-1024, 0), Q = (-8, 4):T = (-0.970066 , 0.242842) , R = (7.000894 , 0.244735)$$
	
	
	\section{Conclusion}

本实验提高精度的主要措施:

1.使用long double 类型;

2.用较多位数来表示$\pi$;

3.将含有除法的方程交叉相乘,通过乘法代替除法,以减少在求商时的误差;

4.在用二分法解非线性方程时,限制条件是结果的绝对值$\leq10^{-7}$。

但由于本实验的方程较为复杂,
含有三角函数、根式、平方等易造成误差放大的因素,
暂时还未找到其他较好的减少误差的办法,但还尝试了另一种思路,叙述如下:

设OT的延长线交PQ于R,则由角平分线定理,有$$\frac{QT}{PT}=\frac{QR}{RP}=\frac{Q_y}{R_y}-1$$
设$T(x,y),k=\frac{Q_y}{Q_x-P_x}$,带入上述方程化简得:
$$\frac{Q_y^2(kx-y)^2+k^2P_x^2y^2-2kQ_yP_xy(kx-y)}{k^2P_x^2y^2}=\frac{Q_y^2+Q_x^2+1-2yQ_y-2xQ_x}{1+P_x^2-2xP_x}$$
故只需用for循环遍历或二分法解非线性方程$$(Q_y^2(kx-y)^2+k^2P_x^2y^2-2kQ_yP_xy(kx-y))(1+P_x^2-2xP_x)-(k^2P_x^2y^2)(Q_y^2+Q_x^2+1-2yQ_y-2xQ_x)=0$$
$$y=\sqrt{1-x^2}$$

由对称性易知
$$R_x=\frac{\frac{Q_xT_y}{T_x} + \frac{T_x(2T_x - Q_x)}{T_y} + 2T_y -2Q_y}{\frac{T_x}{T_y}+\frac{T_y}{T_x}}$$
$$R_y=Q_y-\frac{T_y}{T_x(Q_x - R_x)}$$
但验证结果时发现,上述方法在遇到一些极端情况如$P = (-1.000001, 0), Q = (-2, 2)$时,中间的运算过程会出现极小的浮点数导致无法继续运算,所以这种思路在细节上仍有待改进。

\end{document}